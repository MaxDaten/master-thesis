\documentclass[ngerman]{beamer}

\usepackage{lmodern}
\usepackage[utf8]{inputenc}
\usepackage[T1]{fontenc}
\usepackage[ngerman]{babel}
\usepackage{graphicx}
\usepackage{xcolor}
\usepackage{amsmath}
\usepackage{amsthm}
\usepackage{datetime}

\graphicspath{{./img/}}
\definecolor{halfgray}{gray}{0.55} % chapter numbers will be semi transparent .5 .55 .6 .0


\DeclareMathAlphabet\mathbfcal{OMS}{cmsy}{b}{n}

%% Listings
\usepackage{listings}
\usepackage{sourcecodepro} % now the default typewriter font

\lstset
  { basicstyle=\tiny\ttfamily\footnotesize
  , breaklines=true
  , captionpos=t
  , showstringspaces=false
  , rulecolor=\color{halfgray}
  , commentstyle=\scriptsize
  , numbers=left
  , frameshape={nnY}{n}{n}{Ynn}
  , xleftmargin=2.5em
  , framexleftmargin=2em
  , keywordstyle=\color{blue}
  , backgroundcolor=\color{black!3}
  }

\lstnewenvironment{haskell}[1][]{
    \noindent
    \minipage{\linewidth}
    \vspace{0.5\baselineskip}
    \lstset
        { basicstyle=\footnotesize\ttfamily
        , language=Haskell
        , tabsize=2
        , #1
        }
}{\endminipage}

\usetheme{Warsaw}
% \usetheme{Goettingen}
\usefonttheme{professionalfonts}
\setbeamercovered{transparent}
\setbeamertemplate{footline}[frame number]


\begin{document}
\title[Haskell Engine]{Modern OpenGL Engine in Haskell}
\subtitle[Konzepte]{Funktionale Konzepte und Implementierungen}
\author{Jan-Philip Loos}
\institute[FH Wedel]{Master of Science\\Informatik\\\includegraphics[width=3cm]{fhw}}
\date{\protect\formatdate{02}{06}{2015}}
\maketitle
\logo{\includegraphics[width=0.5cm]{fhw-logo}}

% \frame{\tableofcontents[currentsection]}

\section{Konzept der Renderpipeline}
%% MEALY
\begin{frame}
  \frametitle{Mealy Automat}
  \begin{Definition}
    \begin{align}
    \mathbfcal{M} = \left( Q, \Sigma, \Omega, \delta, \lambda, q_0 \right)
    \label{def:mealy-formal}
    \end{align}
    \begin{align*}
    	\text{mit}\\
    	Q &: \text{Endliche Menge von Zuständen} \\
    	\Sigma  &:\text{Endliches Eingabealphabet} \\
    	\Omega  &:\text{Endliches Ausgabealphabet} \\
    	\delta  &:\text{Zustandsübergangsfunktion}\ Q \times \Sigma \rightarrow Q \\
    	\lambda &:\text{Ausgabefunktion}\ Q \times \Sigma \rightarrow \Omega \\
    	q_0 &: \text{Startzustand}
    \end{align*}
  \end{Definition}
\end{frame}

\begin{frame}[fragile]
  \frametitle{Definition in Haskell}
  \begin{haskell}[label={lst:haskell-mealy},caption={[Definition Mealy in Haskell]Definition Mealy in Haskell\protect\footnotemark}]
    newtype Mealy a b = Mealy {
      runMealy :: a -> (b, Mealy a b)
    }
  \end{haskell}
  \footnotetext{https://hackage.haskell.org/package/machines/}
\end{frame}

\section{Anwendung}
\begin{frame}
  \frametitle{Physically Based Rendering}
  \begin{Definition}
    Eine Definition
  \end{Definition}
\end{frame}


\end{document}
