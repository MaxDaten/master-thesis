\chapter{Überblick über das Rendersystem der Engine}
\label{chap:ueberblick-pipeline}

Auf Grund des Umfangs des Projekts ist es nicht möglich einen all umfassenden Überblick über die Engine zu geben. Die Betrachtung im Detail beschränkt sich im Folgenden auf die entwickelte Renderkomponente der Engine und die dahinter stehenden Konzepte und Überlegungen. Es wird anhand der Quelltextbeispiele demonstriert, dass sich der Umfang der funktionalen Konzepte überschaubar gestaltet. Und dennoch ermöglichen diese Konzepte eine hohe Flexibilität. Sobald die Konzepte verstanden sind, erhöht die Kompaktheit dieser Konzepte die Lesbarkeit und erleichtert den Zugang zu strukturellen ähnlich gelagerten Problemen.

\section{Das Rendersystem}

% anforderung
Moderne Renderverfahren, wie zum Beispiel Deferred Rendering, bestehen aus mehreren einzelnen Renderschritten. In der Regel erzeugt jeder Renderschritt Daten, oft in Form von Texturen oder Buffern. Die Daten dienen den nachfolgenden Renderschritten als Basis für weitere Berechnungen. Daraus ergibt sich ein Abhängigkeitsnetzwerk aus Eingabe und Ausgaben einzelner Renderschritte. Ein Renderschritt kann wiederum aus kleineren Renderschritten zusammengesetzt sein.

Bei der Entwicklung der Renderkomponente wurde das Ziel verfolgt, dass sich die einzelen Renderschritte möglichst komponierbar anwenden lassen. Zusätzlich sollen die einzelen Renderschritte die Möglichkeit erhalten ihren inneren Zustand selbst zu verwalten und Implementierungsdetails zu verbergen. Auf Kompositionsebene sollen Implementierungsdetails möglichst vermeidbar sein.

Die Konzeption des Renderystems startet mit einer formal abstrakten Definition auf Basis eines Mealy Automatens. Die formale Definition wird mit weiteren abstrakten funktionalen Konzepten schrittweise erweitert. Kleine praxisnahe Beispiele demonstrieren, wie die Erweiterungen die Flexibilität und Komponierbarkeit der Renderschritte erhöhen.

\subsection{Mealy als formale Basis}

Das Rendersystem basiert auf dem Konzept eines Mealy Automatens \parencite{Mealy1955}. Ein Mealy Automat ist ein \acl{FST}, ein spezieller endlicher Automat, der nicht nur seine Eingabesprache vom Eingabeband akzeptiert, sondern auch eine Ausgabe auf einem Ausgabeband erzeugt. Ein Transduktor übersetzt die Eingabesprache in eine Ausgabesprache. Bei dem Mealy Automaten, als spezielle Form eines \ac{FST}, definiert sich die Ausgabe über den momentanen Zustand und die Eingabe auf diesem Zustand. Formal lässt sich ein Mealy Automat $\mathbfcal{M}$ als 6-Tupel wie folgt definieren:

\begin{definition}
\begin{align}
\mathbfcal{M} = \left( Q, \Sigma, \Omega, \delta, \lambda, q_0 \right)
\label{def:mealy-formal}
\end{align}
\begin{align*}
	\text{mit}\\
	Q &: \text{Endliche Menge von Zuständen} \\
	\Sigma  &:\text{Endliches Eingabealphabet} \\
	\Omega  &:\text{Endliches Ausgabealphabet} \\
	\delta  &:\text{Zustandsübergangsfunktion}\ Q \times \Sigma \rightarrow Q \\
	\lambda &:\text{Ausgabefunktion}\ Q \times \Sigma \rightarrow \Omega \\
	q_0 &: \text{Startzustand}
\end{align*}
\end{definition}

Die Definition ließe sich noch um die Menge der finalen Endzustände $F \subseteq Q$ hin zu einem 7-Tupel erweitern. In dem Konzept wird auf definierte Endzustände verzichtet, da prinzipiell jede Eingabe akzeptiert wird. Nicht akzpetierbare Eingaben stellen externe Ausnahmefälle (Exceptions) dar. Ein vereinfachtes Beispiel für eine nicht akzeptierbare Eingabe ist die Eingabe $\bot$, nach der unser gesamtes System in einen undefinierten Zustand $\bot$ wechselt.

\paragraph{Abstrakte Definition in Haskell}
\label{sec:abstrakte-definition-haskell}

Ein Mealy Automat lässt sich in Haskell abstrakt wie in \fref{lst:haskell-mealy} definieren. {\ttfamily a} beschreibt die Eingabe als Äquivalent zum Eingabealphabet $\Sigma$ der formalen Definition und analog beschreibt |b| die Ausgabe als Äquivalent zum Ausgabealphabet $\Omega$. Die Zustandsübergangsfunktion $\delta$ und die Ausgabefunktion $\lambda$ ergibt sich aus der jeweils konkreten Implementierung des Mealy. Die Menge der Zustände $Q$ wird jeweils von der konkreten Mealy Implementierung gekapselt, wie in \fref{lst:state-mealy-beispiel} anhand eines Beispiels veranschaulicht wird.

Im Gegensatz zur formalen Definition eines Mealy Automatens lässt sich feststellen, dass die in Haskell gewählte Form des Mealy Automatens weit mehr als nur ein endlicher Automat ist. Die Mächtigkeit des Ein- und Ausgabealphabets ist in der Haskell-Definition nicht beschränkt. Beispielsweise lässt sich ein Haskell-Mealy Automat als Zählmaschine umsetzen. Die Maschine akzeptiert die Menge der natürlichen Zahlen (abzählbar unendlich) und zählt zum Beispiel die geraden Zahlen. Daraus folgt, dass bei dem Haskell-Mealy Automaten die Menge der Zustände unendlich sein kann.

\begin{haskell}[label={lst:haskell-mealy},caption={[Definition Mealy in Haskell]Definition Mealy in Haskell\protect\footnotemark}]
newtype Mealy a b = Mealy { runMealy :: a -> (b, Mealy a b) }
\end{haskell}
\footnotetext{https://hackage.haskell.org/package/machines-0.4.1/docs/Data-Machine-Mealy.html}

Für diese Definition einer Meleay existieren einige nützliche Klassen. Die Klassen werden an dieser Stelle übersprungen, da 
später für die erweiterte monadische Definition eigene Implementierungen entwickelt werden (\fref{sec:konkret-rendersystem}).

Allgemein lässt sich die Funktionsweise von |Mealy| wie folgt beschreiben:\\
|runMealy| führt mit einer Eingabe vom Typ |a| den |Mealy| aus und erzeugt eine Ausgabe vom Typ |b| und einen neuen Mealy-Automaten, der die selbe Eingabesprache in die selbe Ausgabesprache übersetzt. Im einfachsten Fall ist der neu erzeugte Mealy Automat exakt der gleiche aus dem vorherigen Schritt. Ist dies für das gesamte Eingabealphabet gegeben, gilt der Automat als zustandslos.

Für die Konstruktion eines Mealy Automaten mit lokalem Zustand folgt aus der formalen Definition \ref{def:mealy-formal} die Notwendigkeit einer Zustandsübergangsfunktion $\delta$ und einem Startzustand $q_0$. Hiefür definieren wir uns in \fref{lst:state-mealy-ctr} die Hilfsfunktion |localStateMealy|. Der erste Parameter |s -> a -> (b, s)| beschreibt die Kombination der beiden Funktionen $\delta$ und $\lambda$, der zweite Parameter den Startzustand $q_0$.

\begin{haskell}[label={lst:state-mealy-ctr},caption={[Konstruktion Mealy mit lokalem Zustand]Hilfsfunktion zur Konstruktion eines Mealy-Automatens mit lokalem Zustand}]
localStateMealy :: (s -> a -> (b, s)) -> s -> Mealy a b
localStateMealy stateTransition initState = run initState where
  run state = Mealy (\input -> 
    let (output, state') = stateTransition state input 
    in (output, run state'))
\end{haskell}

Mit Hilfe der Funktion |localStateMealy| wird ein Mealy Automat erzeugt, der bei jeder Ausführung die |stateTransition| Funktion auf die aktuellen Eingabe |input| und den momentanen Zustand |state| anwendet, um eine Ausgabe |output| und einen neuen Zustand |state'| zu erzeugen. |output| und der rekursiv mit dem neuen Zustand erzeugte Meleay-Automat werden als Resultat zurück geliefert.

In \fref{lst:state-mealy-beispiel} konstruieren wir einen einfachen Mealy-Automaten der ganzzahlige Eingaben akkumuliert und ausgibt. Abschließend wird in \fref{lst:state-mealy-ausfuehrung} ein Ausführungsbeispiel für |sumMealy| demonstriert.

\begin{haskell}[label={lst:state-mealy-beispiel},caption={[Beispiel Mealy Automat mit lokalem Zustand]Beispiel Mealy Automat mit lokalem Zustand}]
sumMealy :: Mealy Int Int
sumMealy = localStateMealy sumInState 0 where
	sumInState state input = (state, state + input)
\end{haskell}

\begin{haskell}[label={lst:state-mealy-ausfuehrung},nolol,caption={Ausführung Mealy Automat mit lokalem Zustand}]
> let m0 = sumMealy
> let (a,m1) = runMealy m0 5
> a
0
> let (b,m2) = runMealy m1 8
> b
5
> let (c,m3) = runMealy m2 0
> c
13
\end{haskell}

\subsection{Konkrete Definition des Rendersystems}
\label{sec:konkret-rendersystem}

In diesem Abschnitt wird die Anforderungen an das Rendersystem konkretisiert. Aus der konkreten Anforderungen wird die Notwenigkeit abgleitet, die bisherige Haskell Definition der Mealy zu erweitern. Abschließend werden eine Reihe funktionaler Konzepte für die erweiterte Version des Mealy Automatens entwickelt. Es wird aufgezeigt, wie diese Konzepte bei der Komposition der Mealy Automaten helfen, und wie sich die kleinen Bausteine zu einem komplexeren Rendersystem zusammensetzen lassen.

\subsubsection{Informelle Spezifikation}
Ein konkretes Rendersystem soll sich aus mehreren Renderschritten, gennannt Renderpasses\footnote{oder Passes, Singular Pass}, zusammensetzen. Durch die rekursive Definition kann ein Renderpass wiederum ein beliebig komplexes Rendersystem darstellen, so dass zwischen Rendersystem und Renderpass nur semantisch unterschieden wird. Jeder konkrete Pass soll einen konkret definierten Eingabetypen und einen konkret definierten Ausgabetypen besitzen. Ein Renderpass übersetzt die Eingabe in eine Ausgabe und kapselt alle dafür notwendigen Zustände und Vorgänge. Zustandsveränderungen sollen nicht direkt von außen vorgenommen werden können. Zu dem internen Zustand eines Renderpasses gehören beispielsweise Ressourcen wie Framebuffer oder Shader. Der interne Zustand soll nur vom Renderpass selbst sichbar gemacht werden können, und dass auch nur, wenn (ein Teil-) Zustand vom Renderpass in die Ausgabe geleitet wird. Direkte Veränderungen des internen Zustands sind von vornherein ausgeschlossen, da in Haskell Daten unveränderlich\footnote{ausgenommen über Konstrukte wie \texttt{MVars} oder \texttt{IORefs}} sind.

Die Definition des Mealy-Typens aus \fref{lst:haskell-mealy} dient als gedankliche Grundlage für die folgende Definition des Rendersystem, da mit \ref{lst:state-mealy-ctr} gezeigt wurde, dass sich Zustände kapseln lassen. Aus den Anforderungen an das Rendersystem \textit{OpenGL}-Methoden aufzurufen zu können ergibt sich, dass das Konzept des Mealy-Typen noch um einen monadischen Basistypen erweitert werden muss. Wird eine |IO| basierte Monade als monadischer Basistyp verwendet, erlaubt das die Ausführung von |IO| Operationen innerhalb der Mealy. Damit die Mealy auf keinen konkreten monadischen Typen festlegen werden muss, wird das Rendersystem als monadischer Transformator (Monad-Transformer) konstruiert. Die Form des monadischen Transformators erlaubt es, Operationen aus einer anderen Monade, der Basismonade, in die Ausführung von |Mealy| zu heben (lifting). So kann das Rendersystem, beispielsweise über eine |Writer|-Monade, die Fähigkeit erhalten, Vorgänge zu protokollieren und über üine |Reader|-Monade können sich die Renderschritte eine unveränderliche Umgebung (Environment), zum Beispiel als Konfiguration, teilen. Durch das sogenannte `Stacken' von Monaden ist es möglich die Eigenschaften der unterschiedlichsten Monaden unter einer Monade zusammenzufassen, es entsteht ein Monad-Stack, mit allen kombinierten Eigenschaften jeder Monade. Werden zum Beispiel die |ReaderT|\footnotemark und |WriterT|\footref{note:reader-writer-trans} Monaden gestackt, entsteht eine Monade die sowohl eine unveränderliche Umgebung bereitstellt und als auch Vorgänge protokollieren kann\footnote{http://en.wikibooks.org/wiki/Haskell/Monad\_transformers}.

\footnotetext{\label{note:reader-writer-trans}Monadische Transformater Version von \texttt{Reader} bzw. \texttt{Writer}}

\subsubsection{Definition}

Aus den oben gegebenen Anforderungen kann jetzt die Definition für |RenderSystem| in \fref{lst:definition-rendersystem} formuliert werden. |m| entspricht der Basismonade, |i| ist der Typparameter für die Eingabe in das Rendersystem und respektive |o| der Typparameter für die Ausgabe. Das Ausführen eines |RenderSystems| liefert, im Gegensatz zum Mealy Automaten, als Ergebnis eine monadische Operation mit der Ausgabe |o| und dem neuen |RenderSystem| für die nächste Ausführung. In \fref{lst:rendersystem-beispiel-reader} wird exemplarisch ein einfaches zustandsloses |RenderSystem| auf Basis der |Reader|-Monade definiert und in \fref{lst:rendersystem-ausfuehrung-beispiel} die Funktionsweise demonstiert. Das |RenderSystem| gibt den momentanen Integerwert der Umgebung auf stdout aus und liefert den Eingabewert unverändert zurück. Der Integerwert der Reader-Monade könnte zum Beispiel einen globalen Frame-Counter repräsentieren.

\begin{haskell}[label={lst:definition-rendersystem},caption={Definition Rendersystem}]
newtype RenderSystem m i o = RenderSystem { runRenderSystem :: i -> m (o, RenderSystem m i o) }
type RenderPass = RenderSystem -- nur ein semantischer Alias
\end{haskell}

\begin{haskell}[label={lst:rendersystem-beispiel-reader},caption={Beispiel Rendersystem mit ReaderT IO als Basismonade}]
type PrintPass = RenderPass (ReaderT Int IO)

printFrame :: PrintPass a a
printFrame = RenderSystem (\i -> do
  env <- ask
  liftIO (print (show env))
  return (i,printFrame))
\end{haskell}

\begin{haskell}[label={lst:rendersystem-ausfuehrung-beispiel},caption={Ausführungsbeispiel Rendersystem}]
> (a,r0) <- runReaderT (runRenderSystem printFrame "Hallo") 0
"0"
> a
"Hallo"
> (b,r1) <- runReaderT (runRenderSystem r0 "Welt!") 1
"1"
> b
"Welt"
\end{haskell}

Da in der Praxis häufiger unterschiedliche Formen von |RenderPass|es konstruieren werden müssen, werden in \fref{lst:renderpass-ctr} drei Hilffunktionen für die Konstruktion der drei wesentlichen Ausprägungen definiert:

\begin{enumerate}
\item zustandslos und unveränderlich
\item zustandsbehaftet unveränderlich
\item voll dynamisch
\end{enumerate}

\begin{haskell}[label={lst:renderpass-ctr},caption={Konstruktoren für einen RenderPass}]
-- 1. zustandslos und unveraenderlich
mkStaticRenderPass :: Monad m => (i -> m o) -> RenderPass m i o
mkStaticRenderPass f = r where r = RenderSystem (liftM (,r) . f)

-- 2. zustandsbehaftet und unveraenderlich
mkStatefulRenderPass :: Monad m => (s -> i -> m (o,s)) -> s -> RenderPass m i o
mkStatefulRenderPass f = go where
  go s = RenderSystem (\i -> do
    (o, t) <- f s i
    return (o, go t))

-- 3. voll dynamisch
mkDynamicRenderPass :: (i -> m (o, RenderPass m i o)) -> RenderPass m i o
mkDynamicRenderPass = RenderSystem
\end{haskell}


%%%
% Klassen
%%%

\subsubsection{Klassen-Instanzen}
\label{sec:rendersystem-klassen-instanzen}

In \fref{sec:abstrakte-definition-haskell} wurde bereits erwähnt, dass sich für den Mealy-Typen nützliche Klassen-Instanzen implementieren lassen. Die konkreten Implementierungen für den Mealy-Typen wurden übersprungen aber in diesem Kapitel werden äquivalente Implementierungen für unseren Typen |RenderSystem| entwickelt. Aus didaktischen werden nur die jeweils minimal notwendigen Definitionen angegeben, vollständige Definition finden sich im Projekt-Modul |Yage.Rendering.RenderSystem|.

Praktische Anwendungen werden in \fref{chap:anwendung} gegeben, doch eine umfassende Übersicht kann auf Grund des Umfangs nicht gegegeben werden. Umfassende Beispiele finden sich ebenso im Projekt beispielsweise im Modul \linebreak|Yage.Rendering.Pipeline.Deferred|.


%%%
% Functor
%%%

\paragraph{Functor} 
Funktoren, im Folgenden Functor genannt, beschreiben in der Kategorientheorie einer strukturerhaltene Abbildung (Homomorphismus) zwischen zwei Kategorien. In Haskell ist die Klasse der Functoren wie in \fref{lst:class-functor} definiert. Das erste intuitive Verständnis eines |Functor|s in Haskell ist eine Containerstruktur, deren Inhalt von einem Typen |a| auf den Typen |b| überführt werden kann, ähnlich dem |map| für die Listen. Zur Klasse der Functoren gehören darüber hinaus noch weit mehr Strukturen. Zum Beispiel lassen sich auch die Ergebnisse von Funktionen homomorph in eine andere Kategorie überführen ohne die Struktur der Berechnung zu verändern.

\begin{haskell}[label={lst:class-functor},caption={Functor Klasse\protect\footnotemark},nolol]
class Functor f where
  fmap :: (a -> b) -> f a -> f b 
\end{haskell}
\footnotetext{http://hackage.haskell.org/package/base-4.7.0.2/docs/Data-Functor.html}

\Fref{lst:rendersystem-functor} definiert eine Functor Instanz für den |RenderSystem| Typen. Beschrieben wird die Überführung des Ergebnistypens |o| in einen neuen Ergebnistypen |c| (siehe konkrete Signatur für |fmap|). Die Implementierung verdeutlicht, dass die Struktur der Berechnung nicht verändert wird. Es wird lediglich der Ausgabetyp des \linebreak|RenderSystem| in einen neuen Ausgabetyp überführt. Diese Functor Instanz ermöglicht es beispielsweise den Ausgabetypen eines |RenderPass|es, mit Hilfe einer Tranformationsfunktion |f| als Adapter, in den Eingabetypen eines nachfolgenden \linebreak|RenderPass|es zu überführen.

\begin{haskell}[label={lst:rendersystem-functor},caption={Functor Instanz für RenderSystem}]
instance Monad m => Functor (RenderSystem m i) where
  fmap :: (o -> c) -> RenderSystem m i o -> RenderSystem m i c
  fmap f (RenderSystem sys) = RenderSystem (sys >=> \ (o,sys') -> return (f o, fmap f sys'))
\end{haskell}


%%%
% Applicative
%%%

\paragraph{Applicative}
Applikative, im Folgenden Applicatives, sind eine spezielle Ausprägung von Functoren. Ein Applicative wird über zwei Operationen definiert: |pure| bettet einen Wert in den Kontext |f| ein und der Sequenz-Operator |(<*>)| wendet eine Berechnung im Kontext |f| auf einen Wert in |f| an und kombiniert das Resultat \parencite[Kapitel~2]{Paterson2008}. Applicative sind weniger mächtig als Monaden, da sie nur das Sequenzieren von kontextfreien Operationen erlauben. Sie werden gerne dazu verwendet kleine Bausteine als Kombinatoren zu bauen, um sie kontextfrei aufeinander anzuwenden. In Haskell sind Applicatives wie in \fref{lst:class-applicative} definiert.

\begin{haskell}[label={lst:class-applicative},caption={Applicative Klasse\protect\footnotemark},nolol]
class Functor f => Applicative f where
  pure :: a -> f a
  (<*>) :: f (a -> b) -> f a -> f a
\end{haskell}
\footnotetext{http://hackage.haskell.org/package/base-4.7.0.2/docs/Control-Applicative.html}

Für das |RenderSystem| wird in \fref{lst:rendersystem-applicative} die Applicative Instanz definiert. Auch hier wird anhand der Implementierung von |(<*>)| deutlich, dass die in |sysf| eingebettete Funktion auf den in |sysa| eingebetteten Wert ohne Berücksichtung des Kontext angewendet wird. |pure| entspricht einem |RenderSystem|, dass ungeachtet der Eingabe konstant den Wert |b| zurück liefert und unter keinen Umständen das Verhalten wechselt. Mit Hilfe der Appliacative Instanz des |RenderSystem|-Typens können mehrere Rendersysteme zu einer größeren Struktur kombiniert werden, sofern die Rendersysteme nicht von einander direkt abhängen.

\Fref{lst:rendersystem-applicative-beispiel} zeigt an einem Beispiel, wie über die Applicative Instanz ein |RenderPass| |sceneToTexture|  über die Applicative Kombinatoren zu einem größeren |RenderPass| kombiniert wird. Der komponierte |RenderPass| |sceneToCubeTexture|, rendert eine Szene mithilfe sechs Kameras, die jeweils entlag der positiven und negativen 3D Weltachsen ausgerichtet sind. Dynamisch erzeugte Cubemaps werden beispielsweise als Environment-Maps (Umgebungstexturen) verwendet, um dynamische Spiegelungen der Umgebung auf Objekten darzustellen.

\begin{haskell}[label={lst:rendersystem-applicative},caption={Applicative Instanz für RenderSystem}]
instance Monad m => Applicative (RenderSystem m i) where
  pure b = r where r = RenderSystem (return . const (b,r))
  RenderSystem sysf <*> RenderSystem sysa = RenderSystem (\i -> do
    (f, mf) <- sysf i
    (a, ma) <- sysa i
    return (f a, mf <*> ma))
\end{haskell}

\begin{haskell}[label={lst:rendersystem-applicative-beispiel},caption={Applicative RenderSystem Beispiel}]
data Cube a = Cube{left,right,top,bottom,front,back :: a}
-- dummy
data Camera
data Scene
data Texture

sceneToTexture :: Camera -> RenderPass IO Scene Texture
sceneToTexture = ...

scene :: Scene
scene = ...

-- Cube with cameras along the +/- world axis directions
cubeViews :: Cube Camera
cubeViews = ...

-- Renders a scene into a cube map
sceneToCubeTexture :: RenderPass IO Scene (Cube Texture)
sceneToCubeTexture = Cube
	`fmap` sceneToTexture (left cubeViews)
	<*> sceneToTexture (right  cubeViews)
	<*> sceneToTexture (top    cubeViews)
	<*> sceneToTexture (bottom cubeViews)
	<*> sceneToTexture (front  cubeViews)
	<*> sceneToTexture (back   cubeViews)
\end{haskell}


%%%
% Profunctor
%%%

\paragraph{Profunctor}
Während die vorhergehenden Klassen nur Operationen auf dem letzten Argument des Typens definierten, gehört  |RenderSystem| auch noch weiteren Klassen an, die sich für die letzten beiden Argumente definieren lassen. Ein Functor in zwei Argumenten wird auch Bifunctor genannt. Jedes Argument gehört einer eigenen Functor-Kategorie an, die auch voneinander unterschiedlich sein können aber nicht müssen. Zusammmen bilden die beiden Functor-Kategorien ($C$ und $D$) eine Produktkategorie ($C \times D$) \parencite{MacLane1998}.

Ein spezieller Bifunctor ist der Profunctor, dessen erstes Argument contravariant und zweites Argument covariant ist. Ein covarianter Functor entspricht der üblichen Haskell |Functor| Klasse mit einem Morphismus von der Ausgangskategorie hin zur Zielkategorie. Die Contravarianz entspricht dahingehend einem umgekehrten Morphismus, von der Zielkategorie zur Ausgangskategorie. Die Haskell Definition der Klasse |Profunctor| findet sich in \fref{lst:class-profunctor}

\begin{haskell}[label={lst:class-profunctor},caption={Profunctor Klasse\protect\footnotemark},nolol]
class Profunctor p where
  dimap :: (a -> b) -> (c -> d) -> p b c -> p a d
\end{haskell}
\footnotetext{https://hackage.haskell.org/package/profunctors-3.2/docs/Data-Profunctor.html}

Anschaulicher wird die Funktionsweise eines |Profunctor|s wieder bei der Implementierung der |Profunctor| Instanz für das |RenderSystem|. Das zweite covariante Argument des Bifunctors ist der Parameter |o| für den Ausgabetypen des |RenderSystem|s und wird analog zu der Functor Instanz implementiert, mit |g| als Morphismus. Der neue erste Parameter des Bifunctors entspricht dem contravarianten Argument. Der umgekehrte Morphismus |f| wird im Gegenzug auf den Eingabewert angewendet, bevor das entsprechende |RenderSystem| mit der Eingabe ausgeführt wird.

Mit der Profunctor Instanz können wir nicht nur elegant den Ausgabetyp unseres Rendersystems konvertieren, sondern auch auch zu einem allgemein definierten Rendersystem neue Varianten erzeugen die auf unterschiedlichen Eingaben arbeiten, solange wir eine Konvertierungsfunktion für die Eingabe definieren können. Die interne Funktionsweise des Rendersystems bleibt trotz etwaiger Konvertierungen von Ein- und Ausgaben die selbe, es werden lediglich die Schnittstellen einem externen Format angepasst.

\begin{haskell}[label={lst:rendersystem-profunctor},caption={Profunctor Instanz für RenderSystem}]
instance Monad m => Profunctor (RenderSystem m) where
  dimap :: (i -> a) -> (b -> o) -> RenderSystem m a b -> RenderSystem m i o
  dimap f g (RenderSystem sys) = RenderSystem (\i -> do
    (o,sys') <- sys (f i)
    return (g o, dimap f g sys'))
\end{haskell}


%%%
% Category
%%%

\paragraph{Category}

Die |Category| beschreibt die Klasse von Strukturen deren Morphismus sich über beide Argumente komponieren lassen. Die grundlegenste |Category| in Haskell ist die der Funktionen und ihre Komponierbarkeit. Unäre Funktionen |(->)| beschreiben einen Morphismus, der das Argument der Funktion einem Funktionswert zuordnet. Unäre Funktionen sind in Haskell analog zur mathematischen Funktionskomposition komponierbar. Dies wird dadurch ausgedrückt, dass unäre Funktionen der |Category| Klasse angehören. In \fref{lst:class-category} wird die Definition der Klasse ausgeschrieben. Sie besteht aus dem Identitätsmorphismus |id| und der rechts zu links Komposition |(.)|.

\begin{haskell}[label={lst:class-category},caption={Category Klasse\protect\footnotemark},nolol]
class Category cat where
  id :: cat a a
  (.) :: cat b c -> cat a b -> cat a c
\end{haskell}
\footnotetext{https://hackage.haskell.org/package/base-4.7.0.2/docs/Control-Category.html}

Auch |RenderSystem| beschreibt einen Morphismus von der Eingabe |i| zur Ausgabe |o|. Folglich gehört auch |RenderSystem| der Klasse der Category an. \fref{lst:rendersystem-category} listet die Implementierng der |Category| Klasse für |RenderSystem|. 

Der Identitätsmorphismus ist ein |RenderSystem|, dass die Eingabe unverändert ausgibt. Die Komposition ist dadurch implementiert, dass  zuerst die am weitesten rechts stehende Eingabe |a| in das rechte |RenderSystem| |mab| eingeben wird und die entsprechende Ausgabe |b| an das linke |RenderSystem| |mbc| weiter geleitet wird, um die finale Ausgabe |c| zu erhalten. Da die jeweiligen |RenderSysteme| sich dynamisch verändern können, werden die neu erzeugten RenderSysteme ebenso komponiert (mit |mbc' . mab'|). 

Die erlaubt auf elegenate Weise unterschiedliche Stufen eines |RenderSystems| hintereinander zu schalten und zu einem größeren Gesamtsystem zu komponieren. \Fref{lst:rendersystem-komposition-beispiel} zeigt an einem Beispiel, wie ein, durch die Komposition zweier unabhängig von einander definierten Textur-Filter, neuer Filter |bloom| erzeugt werden kann.

\begin{haskell}[label={lst:rendersystem-category},caption={Category Instanz für RenderSystem}]
instance Monad m => Category (RenderSystem m) where
  id = RenderSystem (return . (,id))
  RenderSystem mbc . RenderSystem mab = RenderSystem (\a -> do
    (b, mab') <- mab a
    (c, mbc') <- mbc b
    return (c, mbc' . mab'))
\end{haskell}

\begin{haskell}[label={lst:rendersystem-komposition-beispiel},caption={Beispiel einer Komposition von RenderSystem}]
-- Passes colors with Luma above 'threshold'
lumaFilter :: Float -> RenderSystem IO Texture Texture
lumaFilter treshold = ...

-- Filters 'Texture' with a gaussian-kernel of kernelsize x kernelsize
gaussianFilter :: Int -> RenderSystem IO Texture Texture
gaussianFilter kernelsize = ...

bloom :: RenderSystem IO Texture Texture
bloom = gaussianFilter 64 . lumaFilter 0.4
\end{haskell}


%%%
% Semigroup
%%%

\paragraph{Semigroup}

Eine \textit{Semigroup} bzw. \textit{Halbgruppe} ist eine algebraische Struktur mit einer assoziativen binären Verknüpfung. Sie gilt als Verallgemeinerung eines Monoiden, für die kein neutrales Element benötigt wird und respektive einer \textit{Group} bzw \textit{Gruppe}, für die kein inverses und kein neutrales Element benötigt wird. Die definition der Klasse ist in \fref{lst:class-semigroup} angegeben.

\begin{haskell}[label={lst:class-semigroup},caption={Semigroup Klasse\protect\footnotemark},nolol]
class Semigroup a where
  (<>) :: a -> a -> a -- fuer Monoiden 'mappend'
\end{haskell}
\footnotetext{https://hackage.haskell.org/package/semigroups-0.16.2.2/docs/Data-Semigroup.html}

Die Implementierung in \fref{lst:rendersystem-semigroup} verknüpft zwei |RenderSysteme|, indem beide auf der gleichen Eingabe sequenziell ausführt werden. Die Resultate werden abschließend mit |(<>)| verknüpft.

\begin{haskell}[label={lst:rendersystem-semigroup},caption={Semigroup Instanz für RenderSystem}]
instance (Applicative m, Semigroup o) => Semigroup (RenderSystem m i o) where
	sysX <> sysY =  RenderSystem (\i -> 
		liftA2 (<>) (runRenderSystem sysX i) (runRenderSystem sysY i))
\end{haskell}

\Fref{lst:rendersystem-semigroup-beispiel} ist ein Beispiel dafür, wie die Resulate aus mehreren Render-Schritten zu einer größeren assoziative Struktur zusammengefügt werden können. In diesem Fall wird mehrfach |downsample| auf eine Textur ausgeführt, um anschließend die herunter gerechneten Texturen zu einer Mipmap-Chain zu kombinieren. 

\begin{haskell}[label={lst:rendersystem-semigroup-beispiel},caption={Beispielanwendung Semigroup für RenderSystem}]
import Data.List.NonEmpty
-- | MipmapChain mit einem Basis Element und einer Liste von Mipmap Stufen
-- Instanz von Semigroup
type MipmapChain = NonEmpty

mkBase :: b -> MipmapChain b
mkBase b = b :| []

downsample :: Int -> RenderSystem IO Texture Texture
downsample factor = ...

generateMipmap :: RenderSystem IO Texture (MipmapChain Texture)
generateMipmap = fmap mkBase id 
	<> fmap mkBase (downsample (2^1)) 
	<> fmap mkBase (downsample (2^2)) 
	<> fmap mkBase (downsample (2^3))
\end{haskell}


%%%
% Arrow
%%%

\paragraph{Arrow}

|Arrows| sind eine weitere abstrakte Darstellungsform von Morphismen. Während |Category| die Komponierbarkeit von Morphismen beschreibt, beschreiben Arrows den Morphismus an sich. In Haskell wird ein Arrow über die Operation |arr| und |first| definiert. Zusätzlich gehört zur Arrow Klasse auch die Kompositionsoperation, welche aber schon durch die Spezialisierung von |Category| definiert ist. |arr| stellt den direkten Bezug zu dem grundlegenden Morphismus der unären Funktionen |(->)| und der Arrow Struktur her und hebt einen einfachen Morphismus in den Kontext des |Arrow|s. |first| ist eine Variante, neben dem gespiegelten |second|, der sogenannten Sideline Operation \parencite[Kapitel 1]{Asada2010}. Über |first| kann ein Morphismus |a b c| auf die erste Komponente des Argumentes angewendet werden, während die zweite Komponente unverändert durchgereicht wird. Erwähnenswert ist abschließend, dass jeder |Arrow| auch als |Profunctor| dargestellt werden kann \parencite[Kapitel 3]{Asada2010}.

\begin{haskell}[label={lst:class-arrow},caption={Arrow Klasse\protect\footnotemark},nolol]
class Category a => Arrow a where
  arr :: (b -> c) -> a b c
  first :: a b c -> a (b, d) (c, d)
\end{haskell}
\footnotetext{https://hackage.haskell.org/package/base-4.7.0.2/docs/Control-Arrow.html}


\begin{haskell}[label={lst:rendersystem-arrow},caption={Arrow Instanz für RenderSystem}]
instance Monad m => Arrow (RenderSystem m) where
  arr f = RenderSystem (\i -> return (f i, arr f))
  first sys = RenderSystem (\ (b,d) -> do
    (c, sys') <- runRenderSystem sys b
    return ((c,d), first sys'))
\end{haskell}

%%%
% ArrowChoice
%%%

\paragraph{ArrowChoice}

Vor einem abschließenden Beispiel (\fref{lst:pipeline-vollstaendig}) für die Anwendung der |Arrow| Implementierung unter Verwendung der Arrow-Notation \footnotemark, wird im Folgenden noch für |RenderSystem| die |ArrowChoice| Klasse implementiert, um weitere Möglichkeiten der Arrow Notation nutzen zu können. |Arrow|s die auch |ArrowChoice| implementieren, können in der Arrow Notation |if| und |case| Konstrukte als syntaktischen Zucker verwenden. Die Definition der |ArrowChoice| Klasse ist in \fref{lst:class-arrowchoice} angeben. |left| beschreibt dabei eine Operation die eine Operation auf das Argument anwendet, wenn das Argument mit |Left| markiert wurde. Wurde das Argument mit |Right| markiert, bleibt es unverändert.

\footnotetext{https://downloads.haskell.org/~ghc/7.8.2/docs/html/users\_guide/arrow-notation.html}

\begin{haskell}[label={lst:class-arrowchoice},caption={ArrowChoice Klasse\protect\footnotemark},nolol]
class Arrow a => ArrowChoice a where
  left :: a b c -> a (Either b d) (Either c d)
\end{haskell}
\footnotetext{https://hackage.haskell.org/package/base-4.7.0.2/docs/Control-Arrow.html\#g:5}

Die Implementierung von |ArrowChoice| für unser |RenderSystem| in \fref{lst:rendersystem-arrowchoice} ist trivial. Wurde die Eingabe mit |Left| markiert, wird die Eingabe auf unser |RenderSystem| angewendet und die Ausgabe wird wiederum mit |Left| markiert. Mit |Right| markierte Eingaben passieren das RenderSystem unverändert und ohne Seiteneffekte.

\begin{haskell}[label={lst:rendersystem-arrowchoice},caption={ArrowChoice Instanz für RenderSystem}]
instance Monad m => ArrowChoice (RenderSystem m) where
  left sys = RenderSystem (\case
    Left i  -> do
      (b, sys') <- runRenderSystem sys i
      return (Left b, left sys'))
    Right i -> return (Right i, left sys)
\end{haskell}
