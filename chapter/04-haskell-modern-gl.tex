\chapter{Funktionale Programmierung \& OpenGL}
\label{chap:haskell-modern-gl}

Funktionale Programmierung beschreibtich

\ac{DSA} und die verstärkte Verwendung von Buffern für Daten und Kommands (indirectes Rendering) eröffnet neue Spielräume. Währned häufige Foreign Calls in Haskell zu eigenen Bottlenecks führen können (\warn{Beleg}), helfen die Konzepte generell den CPU Overhead zu reduzieren. Während der CPU Overhead im generellen reduziert wird, reduziert sich zusätzlich der Overhead durch die Reduzierung der Foreign Calls. Der gewonnene Spielraum kann für mehr CPU basierte Aufgaben verwendet werden, aber auch um wartungsunfreundliche Kompromisse zwischen Performance und Code-Hygene. Viele Portierungen von Konsole zu PC erwiesen sich oft als zusätzlich aufwändig, da spezielle Optimierungen auf einzelne Anwendungen die Wartung entsprechender Abschnitte deutlich erschwerten (\warn{BELEG}).
