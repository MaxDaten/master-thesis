\chapter{Aktuelle Engineentwicklung}
\label{chap:engine-uebersicht}

viele interdisziplinäre Bereiche abdecken. Die Anforderungen sind entsprechend denen von anderen großen Softwareprojekten ähnlich. In der Regel soll Software robust und flexibel, wartbar, zugänglich und verständlich sein. Kurze Iterationszyklen sollen dem Entwicklerteam ermöglichen auf die sich verändernde Umwelt und den neuen Anforderungen und Gegebenheiten flexibel zu reagieren. Riesige oft unüberschaubare Features werden nicht mehr im stillen über Jahre entwickelt sondern runtergebrochen und agil im laufenden Entwicklunsprozess.

Moderne Engines zählen inzwischen zu den komplexesten Softwareprojekten. Die Echtzeitanforderung wird immer eine Herausforderung bleiben, da mit neu verfügbarer Resourcen oder frei werdenen Resourcen die Grenze des machbaren verschoben wird.

Moderne Engines richten sicht nicht mehr zwingend ausschließlich an Programmierer, sondern sind inzwischen zu Tools herangewachsen, die es sogar nicht Programmierern erlauben eigene Programmlogiken zu implementieren.

kolaborative unreal engine 5

tools/engines werden nicht mehr einfach am reißbrett entworfen sondern die entwicklung muss dynamisch auf die anforderungen reagieren.
entwicklung niemals abgeschlossen. 
%% Zitat: "Even if the software was perfect when you ship it, it has to adjust, because the world around is changing"
