\chapter{Anwendungsbeispiele}
\label{chap:anwendung}

Es folgt eine exemplarische Implementierung eines Tone-Map Render"-schritts. Die Aufgabe des Render"-schritts besteht darin eine HDR-Textur auf einen vom Monitor darstellbaren RGB Farbraum abzubilden. Die Berechnung der Farbwerte wird im Shader |"res/glsl/sampling/tonemap.frag"| vorgenommen und ist hier, mit einem Verweis auf die Implementierung, nicht weiter aufgeführt. Die folgende Implementierung wird schrittweise erläutert und in Teilen aus didaktischen Gründen etwas gegenüber der eigentlichen Implementierung vereinfacht. Der zusammenhängende Quellcode dieses Beispiels findet sich in \fref{lst:tonemap-pass-vollstaendig}.

\paragraph{Ressourcenverwaltung} In der Implementierung wird eine eigene |Resource| Monade zur Verwaltung der Ressourcen auf Basis des |resourcet|\footnote{https://hackage.haskell.org/package/resourcet} Pakets verwendet. In dieser |Resource| Monade lassen sich Ressourcen akquirieren (zum Beispiel mit |glResource|) und über die |Applicative| und |Functor| Instanzen kombinieren. Verlässt das Programm den Kontext der |Resource| Monade, zum Beispiel beim Beenden der Renderloop, werden alle Ressourcen über die registrierte Operation freigegeben.

\paragraph{StateVars} |StateVars| kapseln zwei IO Operationen, von der eine Operation einen Wert abfragt und und die andere Operation einen Wert setzt. Oft werden StateVars in Verbindung mit externen (foreign) Biblitheken verwendet um veränderliche Zustände zu kapseln. StateVars können als normale Haskell Datenobjekte verwendet werden. Eine Ad-Hoc Implementierung findet sich in \fref{lst:statevar}.

\newpage

\begin{haskell}[label={lst:tonemap-pass-sig},caption={\texttt{ToneMapPass} Signatur},nolol]
type ToneMapInput = (HDRSensor, Texture2D PixelHDR, Maybe (Texture2D PixelHDR))
type ToneMapOutput = (Texture2D PixelRGB8)
type ToneMapPass = RenderPass ResIO ToneMapInput ToneMapOutput
\end{haskell}

Die Eingabe in |ToneMapPass| ist ein Triple aus |HDRSensor|, einer HDR Textur |Texture2D PixelHDR| und einer weiteren optionalen HDR Textur, die additiv mit der ersten im Fragment-Shader gemischt wird. |HDRSensor| ist Teil einer |HDRCamera| und beinhaltet für das Tone-Mapping benötigte Größen, wie den Weiß-Punkt oder die Belichtung (Exposure). Die Ausgabe von |ToneMapPass| ist eine übliche RGB Textur.

\begin{haskell}[label={lst:tonemap-pass-res},caption={\texttt{ToneMapPass} Resourcen Allokation},nolol]
toneMap :: Resource ToneMapPass
toneMap = do
  emptyvao <- glResource
  boundVertexArray $= emptyvao
  fbo <- glResource
  
  pipeline <- [ $(embedShaderFile "res/glsl/sampling/drawRectangle.vert")
              , $(embedShaderFile "res/glsl/sampling/tonemap.frag")]
              `compileShaderPipeline` includePaths
  Just (FragmentShader{..}) <- traverse fragmentUniforms =<< get (fragmentShader $ pipeline^.pipelineProgram)

  outTexture <- liftIO . newIORef =<< createTexture2D GL_TEXTURE_2D (Tex2D 1 1) 1

  return $ mkStaticRenderPass $ \(sensor, sceneTex, mBloomTex) -> do
\end{haskell}

In \fref{lst:tonemap-pass-res}, dem Ressourcenblock des Renderschritts, werden die Ressourcen akquiriert, die zur Ausführung der ab \fref{lst:tonemap-pass-run-resize} beschrieben Routine notwenig sind. Dazu gehören ein Vertex Array Objekt, Framebuffer Objekt und die aus dem Vertex-Shader |drawRectangle.vert| und dem Fragment-Shader |tonemap.frag| erzeugte \textit{OpenGL} Pipeline\footnote{https://www.opengl.org/wiki/GLSL\_Object\#Program\_pipeline\_objects}. Die Uniform Variablen des Frag"-ment-Sha"-ders werden von |fragmentUniforms| als |StateVar|s gekapselt (siehe Verwendung \fref{lst:tonemap-pass-run-pipeline}). Zusätzlich wird eine ausschließlich intern verwendete |IORef| für die Ausgabetextur erzeugt. Da die verwendeten \textit{OpenGL} Texturen\footnote{erzeugt mit \texttt{glTexStorage*}} in ihrer Größe unveränderlich sind, muss für neue Ausgabegrößen eine neue Textur erzeugt werden (siehe \fref{lst:tonemap-pass-run-resize}). |resizeTexture2D| nimmt diese Operation vor, in dem es die benötigten Werte der alten Textur übernimmt und eine neue Textur neuer Größe erzeugt. Die alte Textur wird freigegeben. Dieses Texturobjekt speichern wir uns für den nächsten Aufruf in der |IORef|.

\begin{haskell}[label={lst:tonemap-pass-run-resize},caption={[ToneMapPass Größenanpassung des Framebuffers]\texttt{ToneMapPass} Größenanpassung des Framebuffers},nolol]
    target <- get outTexture
    when (target^.textureDimension /= sceneTex^.textureDimension) $ do
      let V2 w h = sceneTex^.asRectangle.extend
      newtarget <- (\t -> resizeTexture2D t w h) =<< get outTexture
      outTexture $= newtarget
      void $ attachFramebuffer fbo [mkAttachment newtarget] Nothing Nothing
    boundFramebuffer RWFramebuffer $= fbo
\end{haskell}

Die Größenanpassung der Ausgabetextur erzeugt eine neue Textur, so dass die neue Textur noch dem Framebuffer neu angefügt werden muss (\fref{lst:tonemap-pass-run-resize}).

\begin{haskell}[label={lst:tonemap-pass-run-pipeline},caption={\texttt{ToneMapPass} Zuweisung an Uniform Variablen},nolol]
    boundProgramPipeline $= pipeline^.pipelineProgram

    iScene $= sceneTex
    iBloom $= mBloomTex
    iHdrSensor $= sensor
\end{haskell}

In \fref{lst:tonemap-pass-run-pipeline} wird die erzeugte Shaderpipline für diesen Renderschritt global aktiviert und den, in |StateVar|s gekapselten, Uniform Variablen des Fragment-Shaders Werte des Sensors zugewiesen. Zusätzlich werden die Texturen an die jeweiligen Texture-Einheiten von \textit{OpenGL} gebunden. Die Zuweisung ist auch entsprechend in den |StarVar|s gekapselt.

\begin{haskell}[label={lst:tonemap-pass-run-draw-and-out},caption={\texttt{ToneMapPass} Draw-Call und Textur ausgeben},nolol]
    glDrawArrays GL_TRIANGLES 0 3

    get outTexture
\end{haskell}

Abschließend wird der Draw-Call abgesetzt und die nun von \textit{OpenGL} gefüllte Textur als Ausgabe zurück gegeben (\fref{lst:tonemap-pass-run-draw-and-out}).
