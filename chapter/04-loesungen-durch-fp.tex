\chapter{Lösungsansätze durch Funktionale Programmierung}\label{chap:loesungen-durch-fp}

\epigraph{"`In our profession, we desperately need all the help we can get. If a clean shop floor reduces accidents, and well-organized shop tools increase productivity, then I’m all for them."'}{James O. Coplien, aus dem Vorwort zu \spacedlowsmallcaps{Clean Code: A Handbook of Agile Software Craftsmanship}}

Wie schon in \fref{chap:engine-uebersicht} beschrieben, sind Spiele- und Grafik-Engines äußerst komplexe Softwareprojekte. Auch wenn Haskell oft als reine akademische Sprache angesehen wird, findet Haskell inzwischen praktische Anwendung in unterschiedlichen Bereichen. Haskell besitzt dabei einige Vorzüge, die viele andere Sprachen nicht bieten oder nur unter großem Aufwand bieten können.

\section{Statische Typisierung}\label{sec:statische-typisierung}

Die statische Typisierung elimiert eine ganze Kategorie von Fehlerquellen, die sich in untypisierten oder dynamisch typisierten Sprachen zeigen. Die statische Typisierung und das Fehlen von implizieten Seiteneffekten sorgt für eindeutige und klare Signaturen. Auch wenn in der Praxis nicht jede Funktion formal auf ihre Korrektheit bewiesen werden kann, erleichtern klaren Signaturen und die statischen Typen die Ad-hoc Beweisführung (Reasoning), die jeder Entwickler automatisch im Kopf beim Lesen oder Schreiben von Programmen mitführt. Viele Implementierungen von Funktionen ergeben sich oft schon aus der Signatur.

Haskell eliminiert durch das Typ-System viele gängige Fehlerquellen in Software. \texttt{NULL} Referenzen werden vom Erfinder Tony Hoare rückblickend selber als "`[...] my billion-dollar mistake"' bezeichnet\footnote{http://www.infoq.com/presentations/Null-References-The-Billion-Dollar-Mistake-Tony-Hoare}. Sie bilden immer noch eine der größten Fehlerqullen in Software, wenn die Sprache implizte \texttt{NULL} Referenzen erlaubt. 

\begin{quote}
"`Static analysis helps, but NULL problems remain the top fault in our codebase."'\footnote{\label{note:carmack-null}https://twitter.com/id\_aa\_carmack/status/325019679720615936}
\end{quote}

Durch die explizite Kennzeichnung von optionalen Werten durch |Maybe| schließt Haskell kategorisch eine der größten Fehlerquellen in Softwareprojekten noch zum Übersetzungszeitpunkt aus. Generell gilt, dass Fehler, die schon zur Übersetzungszeit auftreten, einen geringeren kritischen Einfluss auf das Entwicklungs- bzw. Produktionssystem haben als Laufzeitfehler. Es ist trivial zu verhindern, dass Kompilierungsfehler überhaupt schon in das Entwicklungssystem gelangen\footnote{Beispielsweise durch ein \textit{Continious Integration} System}. Dahingegend ist es keine Seltenheit, dass Laufzeitfehler den Entwicklunszyklus lange genug unentdeckt überstehen, bis sie in das Produktionssystem gelangen. Das ist ein starkes Argument für die Philosophie, möglichst viele Fehlerquellen zur Kompilierungszeit auszuschließen.

\section{Haskell als Kommunikationsmittel}

Haskell gilt als eine sehr ausdrucksstarke Programmiersprache. Viele funktionale Konzepte erlauben einen neuen und oft höheren Abstraktionsgrad, als zum Beispiel rein Objekt orientierten Sprachen. Funktionale Programmierung erzwingt und begünstigt die Zerlegung von Problemen in kleine Teilprobleme. Kleine Teilprobleme sind einfacher zu verstehen und leichter zu warten. Die klaren ausdrucksstarken Signaturen bilden eine solide Basis für die Kommunikation zwischen den Entwicklern über den Programmcode (siehe \fref{sec:engines-projektstrukturen}).

\section{Komposition}

Die genannten wesentlichen Eigenschaften erlauben in Kombination das Schreiben von wiederverwendbaren und komponierbaren Funktionen.Der Grad der Komponierbarkeit von Systembestandteilen ist eine nicht direkt messbare Größe und lässt sich nicht formal definieren. Intuitiv gilt ein System und seine Bestandteile als gut komponierbar, wenn sich die Bestandteile ohne großen Aufwand frei kombinieren lassen. Durch die Kombination von gut komponierbaren Elementen sollte die Komplexität des Kombinats nicht übermäßig ansteigen. Im Ideal entspricht die Komplexität des Kombinats der Summe der Komplexitäten der Teilkomponenten \parencite[Seite 19]{Blackheath2013}.

Kleine übersichtliche und gut komponierbare Elemente führen zu verständlicheren, zugänglicheren und wartbaren Gesamtsystemen \parencite[Seite 12 ff.]{Stewart2015}. Dies zeigt sich dadurch, dass bereits viele kleine spezialisierte Projekte in Haskell entwickelt wurden, die sich leicht zu einem größeren Projekt zusammensetzen lassen. Die referenzielle Transparenz und das Fehlen von impliziten Seiteneffekten spielen hierbei eine wesentliche Rolle. Zusätzlich ist die Verständigung über klare funktionale bzw. mathematische Konzepte und Regeln eindeutig. Ein |Functor| bleibt ein |Functor| und die Anwendung eines |Functor|s ist überall verständlich. Diese Konzepte erhöhen die Kompositionsfähigkeit und Wiederverwendbarkeit der entwickelten Elemente.

\section{Nebenläufigkeiten}
\label{sec:nebenlaeufigkeiten}

Tim Sweeny von \textit{Epic Games} nennt das "`Shared State Concurrency"' Model aus C++, Java oder C\# eine "`Huge productivity burden"' und weiter sagt er: "`Purely Functional is the right default"'\footnote{\cite[Vgl.][Seite 42 u. Seite 56]{Sweeney2006}\label{note:sweeney-mainstream}}. John Carmack, Gründer von \textit{id Software} und nun CTO bei \textit{Oculus VR}: "`[...] banning mutable shared state. Easier said than done, of course."'\footref{note:carmack-null}.

Die Veränderlichkeit von Daten (mutable) stellt ein großes Hindernis für elegante Nebenläufigkeit in den genannten Sprachen dar. Hinzu kommt, dass Funktionen bzw. Methoden in den genannten Sprachen keine Garantie geben können, dass sie auf ihren Daten garantiert seiteneffektsfrei arbeiten. Das macht es aufwändig nebenläufige System zu entwickeln, da Nebenläufigkeit mit einem großen Synchronisierungs- und Verwaltungsaufwand verbunden ist.

\begin{quote}
"`In a concurrent world, imperative is the wrong default!"'\footref{note:sweeney-mainstream}
\end{quote}

% Neben den grundsätzlichen Gegebenheiten für elegante Nebenläufigkeit in Haskell, bietet Haskell STM
