\chapter{Physically Based Rendering}
\label{chap:pbr}

\acl{PBR} beschreibt beschreibt ein relativ neues Oberflächenmaterial- und Beleuchtungskonzept in der Spieleindustrie. Es gibt dabei kein festes Regelwerk vor und ist daher vielmehr als ein Paradigma zu verstehen, das erlaubt die Wechselwirkung von Licht, Oberflächen und Betrachter allgemeingültig und akkurat zu simulieren \parencite[Kapitel 1]{Rousiers2014}. Es führt dabei auch kein weiteres Beleuchtungsmodell ein, sondern lässt sich mit unterschiedlichen Approximationen der BRDF nutzen. Nicht desto trotz bedeutet die Umstellung auf ein \ac{PBR} Verfahren eine komplette Umstellung der Produktions- und Renderpipeline. In diesem Kapitel geben wir einen kurzen Überblick über die Prinzipien von \ac{PBR} und Beweggründe, warum es sich lohnen könnte, dieses Konzept pratisch einzusetzen.

\section{Warum dieses Verfahren?}
\label{sec:pbr-warum}
bla

\section{Was sind die Grundlagen?}
bla

\section{Wie wird es umgesetzt?}
bla

\section{Wo wird es eingesetzt?}
bla
