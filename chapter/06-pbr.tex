\chapter{Physically Based Rendering}
\label{chap:pbr}

\acl{PBR} beschreibt ein relativ neues Oberflächenmaterial- und Beleuchtungskonzept in der Spieleindustrie. Es basiert auf dem von Disney vorgestelltem Beleuchtungsmodell \parencite{Burley2012}. Während Disney ein festes Beleuchtungsmodell entwickelte, gibt \ac{PBR} kein festes Regelwerk vor und ist daher vielmehr als ein Paradigma zu verstehen, das erlaubt die Wechselwirkung von Licht, Oberflächen und Betrachter allgemeingültig und akkurat zu simulieren \parencite[Kapitel 1]{Rousiers2014}. Es führt dabei auch kein weiteres Beleuchtungsmodell ein, sondern lässt sich mit unterschiedlichen Approximationen der BRDF nutzen. Nicht desto trotz bedeutet die Umstellung auf ein \ac{PBR} Verfahren eine komplette Umstellung der Produktions- und Renderpipeline. In diesem Kapitel geben wir einen kurzen Überblick über die Prinzipien von \ac{PBR} und Beweggründe, warum es sich lohnen könnte, dieses Konzept praktisch umzusetzen.

\section{Gründe für \ac{PBR}}
\label{sec:pbr-warum}
bla

\section{Grundlagen}
bla

\section{Umsetzung}
bla

\section{Wo wird es eingesetzt?}
Inzwischen wird das Verfahren in allen großen Spieleengines eingesetzt: CryEngine \parencite{Schulz2014}, Unreal Engine 4 \parencite{Martin2012}, Frostbite \parencite{Lagarde2014} und der Engine hinter den Call of Duty Titeln \parencite{Lazarov2011}.
