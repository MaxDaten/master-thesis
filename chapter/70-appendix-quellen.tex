\chapter{Quellen}

\section[ToneMapPass]{\texttt{ToneMapPass}}
\label{sec:src-tonemappass}

\begin{haskell}[label={lst:tonemap-pass-vollstaendig},caption={[ToneMapPass vollständig]\texttt{ToneMapPass} vollständig}]
type ToneMapInput = (HDRSensor, Texture2D PixelHDR, Maybe (Texture2D PixelHDR))
type ToneMapOutput = (Texture2D PixelRGB8)
type ToneMapPass = RenderPass ResIO ToneMapInput ToneMapOutput

toneMap :: Resource ToneMapPass
toneMap = do
  emptyvao <- glResource
  boundVertexArray $= emptyvao

  pipeline <- [ $(embedShaderFile "res/glsl/sampling/drawRectangle.vert")
              , $(embedShaderFile "res/glsl/sampling/tonemap.frag")]
              `compileShaderPipeline` includePaths

  Just (FragmentShader{..}) <- traverse fragmentUniforms =<< get (fragmentShader $ pipeline^.pipelineProgram)

  outTexture <- liftIO . newIORef =<< createTexture2D GL_TEXTURE_2D (Tex2D 1 1) 1
  fbo <- glResource
  return $ mkStaticRenderPass $ \(sensor, sceneTex, mBloomTex) -> do
    target <- get outTexture
    when (target^.textureDimension /= sceneTex^.textureDimension) $ do
      let V2 w h = sceneTex^.asRectangle.extend
      newtarget <- (\t -> resizeTexture2D t w h) =<< get outTexture
      outTexture $= newtarget
      void $ attachFramebuffer fbo [mkAttachment newtarget] Nothing Nothing

    boundFramebuffer RWFramebuffer $= fbo

    glDisable GL_DEPTH_TEST
    glDepthMask GL_FALSE
    glDepthFunc GL_ALWAYS
    glDisable GL_BLEND
    glDisable GL_CULL_FACE
    glFrontFace GL_CCW

    boundVertexArray $= emptyvao
    boundProgramPipeline $= pipeline^.pipelineProgram

    iScene $= sceneTex
    iBloom $= mBloomTex
    iHdrSensor $= sensor
    
    glDrawArrays GL_TRIANGLES 0 3

    get outTexture
\end{haskell}

\section[StateVar]{\texttt{StateVar}}
\label{sec:src-statevar}

\haskellcode[caption={[Definition StateVar]Definition \texttt{StateVar}},firstline=4]{src/statevar.hs}

\clearpage
\begin{landscape}
\section{Shader Interface}
\haskellcode[caption={Shader-Schnittstelle},label={lst:uniform-beispiel}]{src/uniform-beispiel.hs}
\end{landscape}

\begin{landscape}
\section{Deferred PBR Pipeline Übersicht}
\label{sec:src-pipeline}
\haskellcode[caption={Deferred PBR Pipeline Übersicht}]{src/deferred-pbr-pipeline.hs}
\end{landscape}


