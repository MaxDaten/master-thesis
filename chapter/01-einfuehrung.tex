\chapter{Einführung}

Die offline bzw. nicht echtzeit fähigen Bildsyntheseverfahren waren schon immer richtungsweisend für die Echtzeit Computergrafik. Dem Prinzip geschuldet überragen die offline Verfahren die online Verfahren in Bildqualität und Güte der Beleuchtungssimulation. Zeit spielt in beiden Kategorien eine wichtige Rolle, doch in unterschiedlichen Dimensionen. Die Echtzeitverfahren orientierten sich immer an den Möglichkeiten der Offlineverfahren und es wurde stetig versucht ähnliche visuelle Eindrücke in Echtzeit darzustellen. Viele Nicht-Echtzeit Verfahren haben bereits den Einzug in die Echtzeit Computer Grafik gefunden, auch wenn meist die Effizienz und Plausibitilät der korrekten Simulation vorgezogen werden musste, unter ihnen Verfahren wie zum Beispiel \acf{SSS} \cite{} oder \acf{PBR} \cite{} aber auch \acf{GI}.

\acf{GI} lässt gerenderte Szenen deutlich realistischer und glaubwürdiger wirken, doch ist \ac{GI} auf Grund der Wechselwirkungen zwischen beliebig vielen Punkten im Raum sehr komplex in der Simulation, so dass \ac{GI} in Echtzeit nur approximativ zu erreichen ist. Da im Echtzeitrendering generell eher der plausible visuelle Eindruck im Vordergrund steht, werden entsprechend nur ausreichend effiziente und approximative Verfahren gesucht, im Gegensatz zum Offline Rendering, in dem meist die physicalische Korrektheit und Akkuratheit im Fokus stehen. In den letzten Jahren wurden einige eher stark approximative \ac{GI} Verfahren entwickelt. Unter ihnen klassische \acf{LM}s, modernere \acf{LPV}s oder recht neue \footnote{erstmals vorgestellt in \cite{crassin2011interactive}} und das hier weiter entwickelte Verfahren \acf{SVOGI}. Während einige Verfahren wie \ac{LM}s nur statische Approximationen bieten, sind andere dynamische Verfahren in einigen Situationen immer noch zu teuer oder bieten keine Möglichkeit hoch frequente spekulare Lichtreflexionen abzubilden. Im \fref{chap:uebersicht} werden die gängigsten Echtzeit-Approximationen genauer betrachtet.
