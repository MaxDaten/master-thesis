\chapter{Funktionale Programmierung \& Modern OpenGL}
\label{chap:haskell-modern-gl}

\begingroup
% \setlength\intextsep{0pt}
\section{Anwendung von Haskell mit Modern OpenGL Konzepten}\label{sec:haskell-gl-anwendung}

Die in \fref{sec:overhead-und-flexibilitaet} genannten Erweiterungen wurden entweder mit dem Ziel entwickelt die \textit{OpenGL} \ac{API} zu vereinfachen oder den \ac{API} Overhead zu reduzieren. Wie diese neuen Konzepte mit Haskell harmonieren wird im folgenden beschrieben.

\paragraph{Direct State Access \& Separate Program Objects} Die Verwendung von Buffern für Daten und Kommandos (\ac{MDI}) und der dadurch reduzierte API Overhead soll neue Spielräume für die CPU eröffnen. Dies wird oft damit beworben, dass die CPU wieder "`interessantere"' Aufgaben übernehmen kann, ansatt Zeit mit dem Aufrufen der Grafik-\ac{API} zu verschwenden. Die neuen Spielräumen können aber auch dazu genutzten werden, neue Ansätze zur Produktivitätssteigung zu wählen. Die generelle Reduzierung des Overheads kommt dabei auch Haskell zugute. Und selbst unter der Annahme, dass C++ Programme immer effizienter\footnote{bzgl. der Laufzeit} als Haskell Programme sind, dürfte sich durch den reduzierten Overhead der Spielraum für Haskell ergeben. Und ob diese Annahme überhaupt in einem relevanten Umfang zutrifft, darf bezweifelt werden (siehe \fref{chap:resultat}).

\begin{wrapfigure}{r}{0.5\linewidth}
\centering
\fbox{
\begin{minipage}[t]{0.9\linewidth}
\begin{smaller}
\paragraph{Quine}
Quine\footnote{https://github.com/ekmett/quine/} ist ein kleines Hobby Projekt von Edward Kmett. Aktuell basiert das Projekt auf den neuen \textit{OpenGL} Haskell-Bindings der \texttt{gl} Bibliothek. Die Demo Szenen sind aktuell nur Fragment-Shader Demos von Shadertoy\footnote{https://www.shadertoy.com/}. Langfristiges Ziel von Quine ist laut Edward Kmett eine Bibliothek in Haskell zu erschaffen die eine Sammlung der Best-Practices von Modern OpenGL zugänglich macht. Der Autor dieser Arbeit beteiligt sich an dem Projekt und hat schon kleinere allgemeine Teile aus dem hier beschriebenen Projekt nach Quine portiert.
\end{smaller}
\end{minipage}
}
\end{wrapfigure}

Zusätzlich lässt sich Haskell dazu als Hebel nutzen, die bessere Modularität von \ac{DSA} besser auszuschöpfen. Die \textit{Separate Program Objects} lassen sich dabei schon in \textit{OpenGL} 4.1 als Vorboten von \ac{DSA} betrachten. Wie bereits beschrieben lassen sich über diese \textit{OpenGL} Erweiterung die Uniform Variablen der einzelnen Shader-Stufen direkt über das Program-Objekt setzen. In Verbindung mit einer |StateVar| (siehe Beispielimplementierung \fref{sec:src-statevar}) und wenigen zusätzlichen Definitionen, lassen sich in Haskell elegante Schnittstellendatentypen für Shader definieren, die sich in Zukunft automatisch generieren ließen. \ref{lst:uniform-beispiel} gibt ein Beispiel, wie sich eine Shader-Schnittstelle über funktionale Konzepte zusammensetzen lässt. Ein relevanter Auszug findet sich in \fref{lst:uniform-beispiel-auszug}:

\haskellcode[caption={Shader-Schnittstelle (Auszug)},label={lst:uniform-beispiel-auszug},firstline=67]{src/uniform-beispiel.hs}
\endgroup

Der Beispielcode zeigt, wie die Funktion |fragmentInterface| mit Hilfe der |Applicative| Implementierung der |UniformVar| das Shader Interface modular zusammensetzt. Das werden einzelne Bausteine wie |materialUniform| und |intUniform| komponiert. Details zur |Applicative| Klasse in Haskell finden sich im nächsten \fref{chap:ueberblick-pipeline}. Die Basisbausteine für die GLSL Uniform Basistypen wurden beispielsweise bereits in Quine (siehe Kasten) implementiert.


\paragraph{Vulkan} Mit \textit{Vulkan} sollen neue Möglichkeiten geschaffen werden, die Buffer in Multi-Threading Systemen zu verwenden. Haskell besitzt herausragende Multi-Threading Eigenschaften, wie in \fref{sec:nebenlaeufigkeiten} umrissen wurde. In vielen imperativen Sprachen stellt Multi-Threading immer noch eine besondere Herausforderung dar (siehe \fref{sec:nebenlaeufigkeiten}). Viele Engines verwenden deswegen Multi-Threading nur zögerlich und äußerst vorsichtig. Haskell zeigt, dass viele Grundprinzipien für Nebenläufigkeiten die erste Wahl sind. Konzepte wie |MVar|s oder \ac{STM} könnten das Synchronisieren von Buffer-Objekte über mehrere Threads hinweg vereinfachen.

\section{Grafik-Engines in Haskell}

C bzw. C++ ist der Industriestandard in der Grafik- und Spieleprogrammierung. Alle kommerziell relevanten Spieleengines sind in C++ entwickelt, unter ihnen: \textit{CryEngine}, \textit{Unreal Engine 4}\footnote{\cite{EpicGames2014}} oder \textit{idTech 5}\footnote{idTech 4 wechselte auf C++, idTech 5 basiert auf idTech 4 \parencite{IdSoftware2011}}. Doch gibt es auch Open-Source Engines die in anderen Sprachen implementiert wurden. Die folgende Betrachtung beschränkt sich auf jene Engines, die entweder direkt in Haskell umgesetzt wurden oder deren Bindings auf bestehende Open-Source Engines aus anderen Sprachen basieren. Zusätzlich gibt der Autor zu jedem Projekt eine subjektive Einschätzung über die Probleme der gewählten Lösungen und warum der Autor der Meinung ist, dass eine möglichst rein funktionale Lösung entwickelt werden sollte. Letzter Punkt wird in \fref{chap:resultat} noch genauer ausgeführt.

\subsection{HGamer3D}

\textit{HGamer3D}\footnote{http://www.hgamer3d.org/} basiert auf (nicht vollständigen) Haskell-Bindings zu \textit{Ogre}\footnote{http://www.ogre3d.org/}. Ogre ist eine objektorientierte Open-Source Grafik-Engine geschrieben in C++. Ohne weiter auf die Fähigkeiten der Ogre Engine einzugehen, stellt sich oft die Adaptierunung von imperativen Bibliotheken auf funktionale Konzepte als aufwändig und nicht immer optimal heraus. Die Notwendigkeit Haskell-Bindings zu komplexen C++ \acsp{API} manuell erzeugen zu müssen setzt hohe Hürden.

Auch wenn Ogre als Framework viele Strukturen und Lösungsansätze für gängige Probleme in der 3D Computergrafik und Spieleprogrammierung bereitstellt, ließen sich viele Ansätze auch direkt funktional umsetzen, ohne große und schwerfällige \acsp{API} adaptieren zu müssen. Die Adaption einer C++-Biblithek kann das Ausnutzen vieler Vorteile von Haskell behindern (z.B. Nebenläufigkeit, oder referenzielle Transparenz).

Zusätzlich entstehen durch Bindings zu externen Bibliotheken neue Abhängigkeiten, die oft eine Anpassung der Tool-Chain erfordern, da die Abhängigkeiten nicht in das bestehende Ökosystem passen. Dies erhöht die Komplexität des Gesamtsystems. Die Erfahrung des Autors hat gezeigt, dass externe Bindings oft zu Komplikationen führen, spätestens dann, wenn die Anwendung die Entwicklungsumgebung verlässt. In der Praxis lassen sich aber selten Abhhängigkeiten zu anderen Sprachen komplett vermeiden. Insbesondere in der Grafikprogrammierung mit \textit{OpenGL} werden die eigentlichen Bindings zu der \textit{OpenGL}-API benötigt. Zusätzlich muss ein plattformabhängiger Render-Kontext erzeugt werden. Dessen Erzeugung ist nicht Teil von \textit{OpenGL} und erfordert eine weitere sprachfremde Komponente.

\paragraph{Meinung des Autors} Es sollten möglichst wenige Fremdabhängigkeiten aus anderen Sprachen genutzt werden, leider ist dies nicht immer möglich (Weitere Ausführungen in \fref{sec:probleme-haskell}). Mit den Ogre Bindings wird die eine komplexe schwer funktional bezwingbare \acs{API} (\textit{OpenGL}) mit einer anderen ersetzt. Vorteil ist unbestritten, dass die Engine ausgereift ist und vieles nicht erst neu implementiert werden muss.

\subsection{LambdaCube 3D}

\textit{LambdaCube 3D}\footnote{https://lambdacube3d.wordpress.com/} ist eine in Haskell definierte und mächtige \ac{DSL}, die es erlaubt Grafikanwendungen bis hin zum Shader komplett in Haskell zu formulieren. Da \textit{OpenGL} eine komplexe und unübersichtliche \acs{API} ist, ist die \ac{DSL} entsprechend komplex und unübersichtlich. Zusätzlich basiert das \textit{OpenGL}-Backend noch auf der Version 3.2, sodass viele neue Shader-Möglichkeiten (z.B. Compute Shader ab 4.3) gar nicht abgedeckt sind. Die Dokumentation beschränkt sich auf den Blog und eine handvoll Beispielen. Das erschwert das Erlernen der DSL deutlich.

Auch generell stellt sich bei \textit{OpenGL} die Frage, wie sinnvoll es ist die komplexe \acs{API} in einer anderen Sprache komplett abzubilden. \textit{OpenGL} besitzt viele erlaubte und nicht erlaubten Zustände. Die erlaubten und nicht erlaubten Zustände sind zudem mitunter treiberspezifisch. Hinzu kommen diverse Erweiterungen, die die Verhaltensweise der \acs{API} massiv beeinflussen und gültige und ungültige Zustände hinzufügen oder entfernen können (siehe \fref{sec:vulkan}).

\paragraph{Meinung des Autors} \textit{OpenGL} lässt sich nicht in einem vertretbaren Aufwand komplett abbilden. Der Aufwand wäre ungefähr mit dem vergleichbar, den Grafikkartenhersteller bei der Implementierung ihrer Grafikkartentreiber betreiben (weitere Ausführungen in \fref{sec:vulkan}). Deswegen sollte eine Auswahl der direkten \textit{OpenGL} Bindings getroffen werden um sie punktuell in funktionale Konzepte zu gießen.

% \subsection{Elm}

% \subsection{Gloss}
% Gloss (2d) ist ein schönes Beispiel dafür wie sich mit einem \textit{OpenGL} backend und mit der konzentration auf das wesentliche eine klare funktionale api schaffen lässt die einfach anzuwenden ist.
